\documentclass[a4paper,12pt]{article}

\usepackage[utf8]{inputenc}


\author{Thiago da Costa P. Senff}
\title{Relatorio: Trabalho 1 \\ \large Paralelização do problema do caixeiro viajante usando OMP}


\begin{document}

    \maketitle
    \newpage

    \tableofcontents
    \newpage
    
    \section{Algoritmo Sequencial}
        O Algoritmo sequencial é bastante simples, ele resolve o problema
        do caixeiro viajante na força bruta usando recurção para montar uma arvore onde cada
        folha possui um dos possiveis caminhos e por fim verifica qual é o menor. A implementação do
        algoritmo tem 3 detalhes importantes, primeiro ela faz cortes em caminhos que serão piores que o melhor caminho que ele ja achou anteriormente.
        O algoritmo tambem utiliza uma heuristica que dá preferencia de escolha para a cidade mais proxima
        quando adicionando uma nova cidade no caminho. Por ultimo vale dizer que o algoritmo faz uma implementação bastante
        duvidosa quando verificando quais cidades já estão no caminho, ele faz a varredura do caminho inteiro verificando a existencia da cidade
        isso poderia ser resolvido com um hash, um array com uma posição para cada cidade contendo 0 caso a cidade não esteja no caminho
        e 1 caso a cidade já esteja no caminho, isso tornaria a verificação O(1) e tornaria o algoritmo muito mais eficiente.
        Essa modificação não foi implementada por conta de limitações de testes, pois a entrada paralela ficaria muito rapida e poderia atrapalhar os testes com tempos muito pequenos.

    \section{Estrategia de Parelização}
        A estrategia de Parelização utilizada foi de percorrer a arvore em ramos separados paralelamente, utilizando o comando for do omp
        na primeira iteração recursiva, assim cada thread cuida de arvores de tamanho n-1 paralelamente, essa divisão não é feita na mesma função com a função inicial.
        Uma dificuldade que essa implementação enfrenta é manter o tamanho do caminho minimo para fazer os cortes, sendo necessario uma 
        o uso de critical para o momento onde a distancia minima é verificada e atualizada. Foi considerado paralelizar a arvores em niveis 
        inferiores utilizando tasks mas testes iniciais mostraram que em geral o tempo na minha maquina se manteve similar com alguns casos 
        com variaçoes de 0 a 6 segundos para melhor e para pior, caso hajam mais nucleos de processamento disponiveis essa abordagem poderia se tornar viavel.

    \section{Metodologia de Testes}
        Para o teste foi feito um teste sobre 3 entradas diferentes, para o codigo paralelo o codigo foi rodado usando desde 1 processador até 4, o maximo da maquina, cada teste foi executado 20 vezes e estaremos olhando para a media deles,
        foi garantido tambem que todos os testes tivessem uma duração maior que 10 segundos para diminuit possiveis influencias de processos externos.
        Para os testes, foi inicialmente pensado em fazer uso do pior e melhor caso do algoritmo, essa ideia acabou sendo descartada, o teste de pior caso aumenta o tempo muito rapidamente
        então para que o teste mais rapido tivesse no minimo os 10 segundos de tempo de execução o terceiro teste acabaria levando uma quantidade irrazoavel. Testar usando o melhor caso por outro lado é possivel porém 
        cria uma grande vantagem para a execução sequencial, que ainda se demonstrou mais lenta mas demonstrando um speedup inferior que todos os outros testes iniciais feitos.
        Foi tambem considerado, durante a parte de planejamento, remover a poda de subarvores com respostas piores que o otimo atual
        o que faria os algoritmos tivessem tempo de execução somente baseado no tamanho da entrada. Essa ideia porem é uma grande desvantagem para o algoritmo sequencial, que alem de perder uma de suas 
        poucas otimizações faria com que não fosse necessario usar critical na versão paralela do algoritmo, uma das poucas fontes de possiveis problemas que ele pode ter, além de tornar o algoritmo menos eficiente de forma geral.
        No final a abordagem tomada foi de criar testes que tendem ao pior caso do algoritmo mas que ainda são mais faceis do que o verdadeiro pior caso, para alcançar isso
        simplesmente garantimos que o algoritmo ira escolher uma cidade errada no inicio da arvore devido a heuristica de cidades mais proximas forçando ele a explorar combinações de caminhos errados. 
        \subsection{Especificações da maquina}
    \section{Analise Sequencial}
    Rodando o algo

    \section{Resultados}
    \section{Analise Paralelo}
    \section{Conclusão}




\end{document}